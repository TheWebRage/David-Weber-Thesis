
\label{Main Method}
%\vspace{-7mm}
%\bigskip

\begin{figure}
\begin{lstlisting}[language=C]
/// <summary>
/// Shows a few examples about using the Adaptive Extention methods
/// </summary>
/// <param name="args">Arguments to control what to simplify; (Path to test, name of test, path to testProj, output path)</param>
static void Main(string[] args)
{
	if (args.Length != 4)
	{
		Console.WriteLine("Incorrect arguments\n");
		return;
	} else {
		Console.WriteLine("Using command line arguments");
		testExample = Path.GetFullPath(args[0]);
		testName = args[1];
		testProj = Path.GetFullPath(args[2]);
		outputFilePath = Path.GetFullPath(args[3]);
	}

	SyntaxList<StatementSyntax> testStatementsRaw = Extentions.GetTestStatements(testExample, testName);

	List<StatementSyntax> testStatements = new List<StatementSyntax>(testStatementsRaw);
	Func<List<StatementSyntax>, bool> buildAndCompareTest = BuildAndRunTest;

	Extentions.SetTestStatements(testExample, Path.Combine(outputFilePath, "Original", testName + "_" + Path.GetFileName(testExample)), testName, testStatements);

	List<StatementSyntax> simplifiedStatements = new List<StatementSyntax>();

	try
	{
		// Run algorithm with parameters
		simplifiedStatements = Extentions.FindSmallestFailingInput<StatementSyntax>(testStatements, buildAndCompareTest);
	}
	catch (Exception ex)
	{
		Console.WriteLine(ex.Message);
	}
	finally
	{
		// Revert the original test file back to the original form
		Extentions.SetTestStatements(testExample, testExample, testName, testStatements);
		Console.WriteLine("Reverting the original file.\nHere is the original file");
	}

	Extentions.SetTestStatements(testExample, Path.Combine(outputFilePath, "Simplified", testName + "_" + Path.GetFileName(testExample)), testName, simplifiedStatements);
	Console.WriteLine("Here are the simpified results.");
}
\end{lstlisting}
\caption{Main method in \mytool}
\label{fig:Main1}
%\vspace{-0.5in}
\end{figure}