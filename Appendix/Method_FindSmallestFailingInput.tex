
\label{FindSmallestInput Method}
%\vspace{-7mm}
%\bigskip


\begin{figure}
\begin{lstlisting}[language=C]
/// <summary>
/// Finds the smallest input for the test and input provided for the test to continue to fail
/// </summary>
/// <typeparam name="T">Type of the list in the input</typeparam>
/// <param name="array">Input array for the failing test</param>
/// <param name="compareTestInput">Function to compare the test input against</param>
/// <returns>A list of the smallest failing input for the test to continue to fail</returns>
static public List<T> FindSmallestFailingInput<T>(List<T> array, Func<List<T>, bool> compareTestInput)
{
	int numSections = 2;

	while (true)
	{
		bool isSuccessful = true;
		List<List<T>> sectionedArray = GetDividedSections(numSections, array);

		// Test the sections for failing input
		foreach (List<T> arrSection in sectionedArray)
		{
			isSuccessful = compareTestInput(arrSection);

			if (!isSuccessful && arrSection.Any())
			{
				// Section off failing input and try again
				array = arrSection;
				numSections = 2;
				break;
			}
		}

		if (!isSuccessful)	continue;

		// Test the compliments of the sections for failing input
		foreach (List<T> arrSection in sectionedArray)
		{
			List<T> compliment = GetSectionCompliment(sectionedArray, sectionedArray.IndexOf(arrSection));
			isSuccessful = compareTestInput(compliment);

			if (!isSuccessful && compliment.Any())
			{
				// Section off failing input and try again
				array = compliment;
				numSections = Math.Max(numSections - 1, 2);
				break;
			}
		}

		if (!isSuccessful)	continue;

		// If all previous inputs pass, increase granularity, create more equal parts
		numSections = 2 * numSections;

		if (numSections > array.Count)
		{
			return array;
		}
	}
}

\end{lstlisting}
\caption{FindSmallestFailingInput method in \mytool}
\label{fig:FindSmallestFailingInput1}
%\vspace{-0.5in}
\end{figure}
