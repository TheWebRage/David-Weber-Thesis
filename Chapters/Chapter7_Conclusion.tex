\clearpage % clear the prior chapter's page

\chapter{Conclusion}\label{CH7_Conclusion}
%\vspace{-7mm}
%\bigskip

\section{Further Work}

Even though \mytool shows usefulness and effectiveness, there is still much that can be improved on this tool. One major downside to this approach is the lack of HDD in the program. It will try and parse the unit tests as a flat structure every time. This has the potential to behave poorly and incorrectly for more complex tree structures that may contain loops, conditional statements, or action type statements. However, after the results of this research, a majority of C\# tests are of this type of structure. We can take advantage of this and use DD for this majority of the tests. If we then implement an HDD approach as well, we can benefit from both approaches for this tool. Using DD for flat structures for faster parsing and using HDD for complex trees allows for a greater effectiveness of the simplification. 

Another improvement that can be researched into is performance increases. Simplifying these 30 unit tests using \mytool took a while to run. If this needed to execute for a project such as the language-ext project with over 2600 tests, with a change that broke even 1\% of the bugs, this would still be 26 unit tests that are needed to simplify. Additionally if this was introduced as a step in a pipeline, then would be expensive to keep running. However, if performance increases were found and implemented, then this would be an even more efficient and effective tool to utilize.

\section{Conclusion}

 After running through the tests and conducting experiments on 30 bugs, \mytool performed well with great results. Most of the tests it was able to parse correctly and even a good number it was able to parse perfectly, leaving only a few tests it was not able to reduce the test while leaving in the failing logic. However, from an initial perspective, implementing another HDD approach alongside this DD approach seems to be the solution for most of these issues. Additionally, it seems current coding standards are a great contributor to the effectiveness of this simple tool. Since C\# coding standards lead developers to write simple unit tests with a flat structure, it allows for the fast, yet simple, DD algorithm to be useful.